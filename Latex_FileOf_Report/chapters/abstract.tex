\begin{abstract}
The segmentation of water bodies in very high-resolution satellite and aerial images is a crucial task in remote sensing, essential for applications such as environmental monitoring, urban planning, and disaster management. Traditional methods often struggle to cope with the intricacies and high variability of these high-resolution images. This thesis addresses these challenges by leveraging the DeepLabv3+ model, a architecture for semantic segmentation, enhanced through the integration of an ensemble of feature extraction backbones.
Our approach employs ResNet50V2, MobileNetV3, and EfficientNetV2 as backbones in the encoder part of the DeepLabv3+ model, utilizing a weighted averaging ensemble technique to combine their strengths. This ensemble method aims to improve the feature extraction process and enhance the overall segmentation performance. The backbones were selected for their complementary characteristics: ResNet50V2 for its robust feature extraction, MobileNetV3 for its computational efficiency, and EfficientNetV2 for its advanced scaling capabilities.
 The results demonstrate significant improvements in key evaluation metrics, with the ensemble model achieving approximately 96\% accuracy and a mean Intersection over Union (mean IoU) of 88\%. Additionally, metrics such as the F1-score, precision, and recall also showed marked enhancements, confirming the effectiveness of the weighted averaging technique in aggregating the outputs of the different backbones.
Our research provides a comprehensive analysis of the benefits of using an ensemble of ResNet50V2, MobileNetV3, and EfficientNetV2 backbones in the DeepLabv3+ model for the segmentation of water bodies in high-resolution remote sensing images. The promising results underscore the potential of ensemble techniques in improving segmentation performance.



\textbf{Keywords:} Water Bodies segmentation, DeepLabV3+, ResNet50V2, remote sensing imagery, Semantic segmentation, Atrous convolution, EfficientNetV2, MobileNetV3.

\end{abstract}

